\documentclass[a4paper,10pt]{article}
\usepackage{eumat}

\begin{document}
\begin{eulernotebook}
\eulerheading{Identitas Diri}
\begin{eulercomment}
Nama : Assyifa Cahya Marsellita\\
NIM  : 22305141001\\
Matematika B 22

\end{eulercomment}
\eulersubheading{Analisis Data Statistika Deskriptif}
\begin{eulercomment}
\end{eulercomment}
\begin{eulerttcomment}
   Menurut Sudjana (2000) statistika adalah pengetahuan yang
\end{eulerttcomment}
\begin{eulercomment}
berhubungan dengan cara-cara pengumpulan data, pengolahan atau
penganalisaannya dan penarikan kesimpulan berdasarkan kumpulan data
dan penganalisaan yang dilakukan. Tujuan dari statistika adalah untuk
menyajikan data secara ringkas dan mudah dimengerti, sehingga dapat
membuat kesimpulan dari suatu populasi dari data yang diambil.
Statistika dibagi menjadi statistika deskriptif dan inferensial.\\
\end{eulercomment}
\begin{eulerttcomment}
   Statistika deskriptif merupakakan cabang statistika yang fokus pada
\end{eulerttcomment}
\begin{eulercomment}
pengumpulan, penyajian, dan menganalisa yang berwujud angka-angka agar
data tersebut dapat ditampilkan secara ringkas dan informatif. Dalam
statistika terdapat empat jenis statistika deskriptif yaitu,
distribusi frekuensi, ukuran pemusatan, ukuran letak, dan ukuran
penyebaran.\\
\end{eulercomment}
\begin{eulerttcomment}
   Distribusi frekuensi adalah merangkum data dalam bentuk tabel atau
\end{eulerttcomment}
\begin{eulercomment}
grafik, yang menunjukkan seberapa sering setiap nilai muncul dalam
suatu himpunan data. Ukuran pemusatan terdiri dari mean (rata-rata),
median (nilai tengah), dan modus (nilai yang sering muncul). Ukuran
letak terdiri dari kuartil (membagi data menjadi empat bagian yang
sama besar), desil (membagi data menjadi 10 bagian sama besar), dan
persentil (membagi data menjadi 100 bagian sama besar). Ukuran
penyebaran terdiri dari jangkauan (nilai maksimum dan minimum dalam
suatu set data), varians (mengukur sejauh mana nilai tersebut menyebar
dari rata-ratanya), simpangan baku (akar kuadrat dari varians, yang
memberikan gambaran tentang seberapa dekat nilai-nilai dalam suatu set
data terhadap rata-ratanya.)\\
\end{eulercomment}
\begin{eulerttcomment}
   Dari penjelasan singkat tentang data statistika deskriptif maka
\end{eulerttcomment}
\begin{eulercomment}
pada sub bab ini, saya akan membahas analisis data statistika
deskriptif dengan jenis ukuran letak dan ukuran penyebaran.
\end{eulercomment}
\eulersubheading{Ukuran Letak}
\begin{eulercomment}
Kita singgung kembali. Ukuran letak disini terdiri dari kuartil
(membagi data menjadi empat bagian yang sama besar), desil (membagi
data menjadi 10 bagian sama besar), dan persentil (membagi data
menjadi 100 bagian sama besar).

\begin{eulercomment}
\eulerheading{Kuartil}
\begin{eulercomment}
Kuartil merupakan penggambaran pembagian data menjadi empat bagian
yang sama besar. Kuartil membagi data menjadi tiga titik tertentu,
yang dikenal sebagai kuartil pertama (Q1), kuartil kedua (Q2), dan
kuartil ketiga (Q3). Rumus yang digunakan dalam menentukan posisi
Q1,Q2,dan Q3 dengan\\
\end{eulercomment}
\begin{eulerformula}
\[
Qi = (i(n+1))/4
\]
\end{eulerformula}
\begin{eulercomment}
dimana i adalah indeks kuartil dan n adalah jumlah total data

Dalam EMT fungsi yang dapat digunakan untuk menentukan kuartil adalah
'quartiles()'.
\end{eulercomment}
\begin{eulerprompt}
>data=[93,80,52,41,60,77,55,71,79,81,64,83,32,95,75,54,90,80,95];
\end{eulerprompt}
\begin{eulercomment}
Diketahui sebuah data dari hasil nilai ujian akhir dari siswa
\end{eulercomment}
\begin{eulerprompt}
>urut=sort(data)
\end{eulerprompt}
\begin{euleroutput}
  [32,  41,  52,  54,  55,  60,  64,  71,  75,  77,  79,  80,  80,  81,
  83,  90,  93,  95,  95]
\end{euleroutput}
\begin{eulercomment}
Dalam menentukan kuartil langkah pertama yang dapat dilakukan dengan
mengurutkan data tersebut dengan fungsi sort(data). Fungsi sort(data)
dalam Euler Math Toolbox digunakan untuk mengurutkan elemen-elemen
dalam suatu vektor atau matriks (dari nilai terkecil ke nilai
terbesar).
\end{eulercomment}
\begin{eulerprompt}
>quartiles(data)
\end{eulerprompt}
\begin{euleroutput}
  [32,  55,  77,  83,  95]
\end{euleroutput}
\begin{eulercomment}
Dalam output hitung yang dihasilkan dari 'quartiles(data)' dapat
diketahui bahwa nilai Q1(kuartil bawah) = 55 , Q2(kuartil
tengah(median)) = 77, dan Q3(kuartil atas)= 83. Lalu untuk nilai
paling kanan dan paling kiri merupakan minimum dan maximum dari suatu
data yang diketahui.

Dengan cara rumus kuartil yang diketahui dapat dihitung nilai kuartil
atas, tengah, dan bawah. Dengan cara...
\end{eulercomment}
\begin{eulerprompt}
>data=[93,80,52,41,60,77,55,71,79,81,64,83,32,95,75,54,90,80,95];
>urut=sort(data)
\end{eulerprompt}
\begin{euleroutput}
  [32,  41,  52,  54,  55,  60,  64,  71,  75,  77,  79,  80,  80,  81,
  83,  90,  93,  95,  95]
\end{euleroutput}
\begin{eulerprompt}
>a=length(data)
\end{eulerprompt}
\begin{euleroutput}
  19
\end{euleroutput}
\begin{eulercomment}
Setelah mengurutkan data tersebut, selanjutnya hitung banyaknya data
tersebut dengan menggunakan fungsi 'length()'. Fungsi tersebut
berfungsi untuk mengetahui banyaknya elemen dalam suatu vektor atau
matriks. 
\end{eulercomment}
\begin{eulerprompt}
>Q1=((a+1)/4)
\end{eulerprompt}
\begin{euleroutput}
  5
\end{euleroutput}
\begin{eulercomment}
Hasil Q1 menunjukkan 5 yang berarti letak kuartil ke-1 ada di data
nomer 5 yaitu 55.
\end{eulercomment}
\begin{eulerprompt}
>Q2=((a+1)/2)
\end{eulerprompt}
\begin{euleroutput}
  10
\end{euleroutput}
\begin{eulercomment}
Hasil Q2 menunjukkan 10 yang berarti letak kuartil ke-2 ada di data
nomer 10 yaitu 77.
\end{eulercomment}
\begin{eulerprompt}
>Q3=60/4
\end{eulerprompt}
\begin{euleroutput}
  15
\end{euleroutput}
\begin{eulercomment}
Hasil Q3 menunjukkan 15 yang berarti letak kuartil ke-3 ada di data
nomer 15 yaitu 83. Mengapa fungsi ini saya hitung langsung? karena
saat percobaan EMT tidak biasa membaca perintah maka dari itu saya
menggunakan EMT secara langsung.


Dari beberapa percobaan tersebut dapat diketahui bahwa dengan cara
rumus kuartil ataupun menggunakan fungsi 'quartiles()' tersebut akan
sama.
\end{eulercomment}
\eulersubheading{Desil dan Persentil}
\begin{eulercomment}
Desil merupakan pembagian data ke dalam sepuluh kelompok sebanding
yang disusun berdasarkan urutan nilainya. Desil ke-1 (D1) adalah nilai
terendah, desil ke-2 (D2) adalah nilai yang membagi data menjadi 10\%
terendah, desil ke-3 (D3) membagi data menjadi 20\% terendah, dan
seterusnya. Dalam EMT fungsi yang digunakan yaitu 'quantile()'


Sedangkan persentil suatu nilai atau titik data yang membagi
distribusi data menjadi persentase tertentu atau menjadi 100 bagian
yang sama besar. Dalam EMT fungsi yang digunakan sama seperti desil
yaitu 'quantile()'. Perbedaan penggunaannya terletak pada nilai yang
akan dibaginya.
\end{eulercomment}
\begin{eulerprompt}
>data=[93,80,52,41,60,77,55,71,79,81,64,83,32,95,75,54,90,80,95,55];
>urut=sort(data)
\end{eulerprompt}
\begin{euleroutput}
  [32,  41,  52,  54,  55,  55,  60,  64,  71,  75,  77,  79,  80,  80,
  81,  83,  90,  93,  95,  95]
\end{euleroutput}
\begin{eulerprompt}
>quantile(urut,0.1)
\end{eulerprompt}
\begin{euleroutput}
  50.9
\end{euleroutput}
\begin{eulercomment}
Dari hasil tersebut dapat diketahui bawha nilai desil ke-1 dan
persentil ke-10 adalah 50,9
\end{eulercomment}
\begin{eulerprompt}
>quantile(urut,0.2)
\end{eulerprompt}
\begin{euleroutput}
  54.8
\end{euleroutput}
\begin{eulerprompt}
>quantile(urut,0.5)
\end{eulerprompt}
\begin{euleroutput}
  76
\end{euleroutput}
\begin{eulerprompt}
>quantile(urut,0.7)
\end{eulerprompt}
\begin{euleroutput}
  80.3
\end{euleroutput}
\begin{eulercomment}
Dari dua percobaan tersebut diketahui nilai persentil ke-20 dan
persentil ke-50 adalah 54,8 dan 76. Hasil tersebut sama dengan hasil
desil ke-2 dan desil ke-5.

Karena desil ke-8 (D8) adalah nilai yang membagi data menjadi 80\% di
bawahnya dan 20\% di atasnya. Sedangkan persentil 80\% (P80) juga
merujuk pada nilai yang membagi data menjadi 80\% di bawahnya dan 20\%
di atasnya.
\end{eulercomment}
\eulersubheading{Ukuran Penyebaran}
\begin{eulercomment}
Kita singgung kembali. Ukuran penyebaran disini terdiri dari jangkauan
yang terdiri dari nilai maksimum dan minimum, varians (mengukur sejauh
mana nilai tersebut menyebar dari rata-ratanya), simpangan baku(akar
kuadrat dari varians yang memberikan gambaran tentang seberapa dekat
nilai-nilai dalam suatu set data terhadap rata-ratanya).

\end{eulercomment}
\eulersubheading{Jangkauan}
\begin{eulercomment}
Jangkauan (range) adalah salah satu ukuran penyebaran yang paling
sederhana dalam statistika. Jangkauan dapat dihitung dengan mengambil
selisih antara nilai maksimum dan minimum dalam suatu set data.

Untuk menemukan jangkauan data tunggal di EMT dapat menggunakan
perintah berikut:\\
\textgreater{} x=[data]; max(x)-min(x)
\end{eulercomment}
\begin{eulerprompt}
>data=[65,55,70,85,90,75,80,75];
>a=max(data)
\end{eulerprompt}
\begin{euleroutput}
  90
\end{euleroutput}
\begin{eulerprompt}
>b=min(data)
\end{eulerprompt}
\begin{euleroutput}
  55
\end{euleroutput}
\begin{eulercomment}
Permisalan dengan a = max(data) dan b = min(data) memudahkan untuk
menghitung nilai jangkauan.
\end{eulercomment}
\begin{eulerprompt}
>jangkauan = a-b
\end{eulerprompt}
\begin{euleroutput}
  35
\end{euleroutput}
\begin{eulercomment}
Dalam menentukan nilai jangkauan antara nilai maximum dan minimum
dapat dilakukan dengan cara menentukan nilai maximum dan minimum dari
data tersebut, lalu kurangi antara hasil dari nilai maximum dan
minimum tersebut.

Untuk data berkelompok dapat menggunakan rumus;\\
\textgreater{} max(transpose(T[,2]))-min(transpose(T[,1]))

Dimisalkan diketahui sebuah table distribusi frekuensi
\end{eulercomment}
\begin{eulerprompt}
>r=39.5:5:69.5; v=[5,18,42,20,9,6];
>T:=r[1:6]' |r[2:7]'| v'; writetable(T,labc=["TB","TA","Frek"])
\end{eulerprompt}
\begin{euleroutput}
          TB        TA      Frek
        39.5      44.5         5
        44.5      49.5        18
        49.5      54.5        42
        54.5      59.5        20
        59.5      64.5         9
        64.5      69.5         6
\end{euleroutput}
\begin{eulerprompt}
>max(transpose(T[,2]))-min(transpose(T[,1]))
\end{eulerprompt}
\begin{euleroutput}
  30
\end{euleroutput}
\begin{eulercomment}
Jadi dapat diketahui bahwa jangkauan dari nilai tersebut adalah 30
orang. 
\end{eulercomment}
\eulersubheading{Varians}
\begin{eulercomment}
Varians adalah nilai statistik yang sering kali dipakai dalam
menentukan kedekatan sebaran data yang ada di dalam sampel dan
seberapa dekat titik data individu dengan mean atau rata-rata nilai
dari sampel itu sendiri.

Pada EMT, dalam menentukan suatu varians dapat menggunakan fungsi
'dev()\textasciicircum{}2'. Fungsi dev disini merupakan kepanjangan dari deviations
yang berarti suatu verktor atau array yang berisi deviasi dari setiap
nilai dalam data. 

\end{eulercomment}
\begin{eulerformula}
\[
{dev} = x_i - \bar{x}
\]
\end{eulerformula}
\begin{eulerprompt}
>data=[65,55,70,85,90,75,80,75];
>urut=sort(data)
\end{eulerprompt}
\begin{euleroutput}
  [55,  65,  70,  75,  75,  80,  85,  90]
\end{euleroutput}
\begin{eulerprompt}
>a = mean(urut)
\end{eulerprompt}
\begin{euleroutput}
  74.375
\end{euleroutput}
\begin{eulerprompt}
>dev = urut-a
>varians = mean(dev^2)
\end{eulerprompt}
\begin{euleroutput}
  108.984375
\end{euleroutput}
\begin{eulercomment}
Jadi nilai varians dari data tersebut adalah 108,984375. Lali
bagaimana jika suatu data tersebut berkelompok?
\end{eulercomment}
\begin{eulerprompt}
>r=499.5:100:1099.5; v=[4,6,12,15,10,3];
>T:=r[1:6]' | r[2:7]' | v'; writetable(T,labc=["tepi bawah","tepi atas","frekuensi"])
\end{eulerprompt}
\begin{euleroutput}
   tepi bawah tepi atas frekuensi
        499.5     599.5         4
        599.5     699.5         6
        699.5     799.5        12
        799.5     899.5        15
        899.5     999.5        10
        999.5    1099.5         3
\end{euleroutput}
\begin{eulerprompt}
>(T[,1]+T[,2])/2; t=fold(r,[0.5,0.5])
\end{eulerprompt}
\begin{euleroutput}
  [549.5,  649.5,  749.5,  849.5,  949.5,  1049.5]
\end{euleroutput}
\begin{eulerprompt}
>m = mean(t,v)
\end{eulerprompt}
\begin{euleroutput}
  809.5
\end{euleroutput}
\begin{eulerprompt}
>sum(v*(t-m)^2)/(sum(v)-1)
\end{eulerprompt}
\begin{euleroutput}
  17551.0204082
\end{euleroutput}
\eulersubheading{Simpangan Baku}
\begin{eulercomment}
Simpangan baku atau deviasi standar adalah ukuran seberapa jauh
nilai-nilai dalam satu set data tersebar dari nilai rata-ratanya. Ini
memberikan gambaran tentang sejauh mana nilai-nilai dalam data
"berkumpul" atau "menyebar" di sekitar rata-ratanya.

\end{eulercomment}
\begin{eulerformula}
\[
\sigma=\sqrt{varians}
\]
\end{eulerformula}
\begin{eulercomment}
Pada sub bab sebelumnya telah dijelaskan bagaiaman cara mencari nilai
suatu varians. Mengulang kembali varians menggunakan fungsi 

\end{eulercomment}
\begin{eulerformula}
\[
mean*dev()^2
\]
\end{eulerformula}
\begin{eulercomment}
sedangkan simpangan baku menggunakan\\
\end{eulercomment}
\begin{eulerformula}
\[
\sigma=\sqrt{mean*dev()^2}
\]
\end{eulerformula}
\begin{eulerttcomment}
 
\end{eulerttcomment}
\begin{eulerprompt}
>data=[65,55,70,85,90,75,80,75];
>urut=sort(data)
\end{eulerprompt}
\begin{euleroutput}
  [55,  65,  70,  75,  75,  80,  85,  90]
\end{euleroutput}
\begin{eulerprompt}
>a = mean(urut)
\end{eulerprompt}
\begin{euleroutput}
  74.375
\end{euleroutput}
\begin{eulercomment}
Menentukan sebuah rata-rata dari data tersebut.
\end{eulercomment}
\begin{eulerprompt}
>dev = urut-a
\end{eulerprompt}
\begin{euleroutput}
  [-19.375,  -9.375,  -4.375,  0.625,  0.625,  5.625,  10.625,  15.625]
\end{euleroutput}
\begin{eulerprompt}
>varians = mean(dev^2)
\end{eulerprompt}
\begin{euleroutput}
  108.984375
\end{euleroutput}
\begin{eulerprompt}
>simpanganBaku = sqrt(var)
\end{eulerprompt}
\begin{euleroutput}
  10.4395581803
\end{euleroutput}
\begin{eulercomment}
Nilai simpangan baku dari data tersebut adalah 10.4395581803
\end{eulercomment}
\begin{eulerprompt}
>r=499.5:100:1099.5; v=[4,6,12,15,10,3];
>T:=r[1:6]' | r[2:7]' | v'; writetable(T,labc=["tepi bawah","tepi atas","frekuensi"])
\end{eulerprompt}
\begin{euleroutput}
   tepi bawah tepi atas frekuensi
        499.5     599.5         4
        599.5     699.5         6
        699.5     799.5        12
        799.5     899.5        15
        899.5     999.5        10
        999.5    1099.5         3
\end{euleroutput}
\begin{eulerprompt}
>(T[,1]+T[,2])/2; t=fold(r,[0.5,0.5]); m=mean(t,v);
>sqrt(sum(v*(t-m)^2)/(sum(v)-1))
\end{eulerprompt}
\begin{euleroutput}
  132.480264221
\end{euleroutput}
\end{eulernotebook}
\end{document}
